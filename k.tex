\documentclass[11pt]{article}
\usepackage{indentfirst}
\usepackage{stmaryrd}
\usepackage{amscd}
\usepackage[T2A]{fontenc}
\usepackage[utf8]{inputenc}
\usepackage[russian]{babel}
\usepackage{amsthm, amsmath, amssymb, stmaryrd}
\usepackage[dvips]{graphicx}
\theoremstyle{definition}
\newtheorem{defin}{Определение}%{section}
\theoremstyle{plain}
\theoremstyle{plain}

\newtheorem{remark}{Замечание}
\newtheorem{problem}{Задача}
\newtheorem{lemma}{Лемма}
\newtheorem{pred}{Предложение}
\newtheorem{theorem}{Теорема}

\def\leq{\leqslant}
\def\geq{\geqslant}
\def\wt#1{\widetilde{#1}}
\def\wtB#1{B_{\wt{x}_0}^{\wt{X}}(#1)}
\def\eps{\varepsilon}

\begin{document}
\section{Объёмная энтропия}
Пусть у нас есть $(X, d)$ - компактное пространство с внутренней метрикой $d$, локально стягиваемое, 
а $\mu$ -- конечная мера на $X$. Рассмотрим универсальную накрывающую -- $\wt{X}$, отображение проекции $p : \wt{X} \rightarrow X$, 
поднятие метрики $\wt{d}$ и меры $\wt{\mu}$.
Зафиксируем точку $\wt{x}_0 \in \wt{X}$ и соответствующую ей точку $x_0 \in X$.
\begin{defin}
Точку пространства $\wt{X}$  будем называть {\it центральной}, если она отображается в $x_0$ при проекции $p$.
\end{defin}

\begin{defin}
	Объёмной энтропией $h(X, d, \mu)$, называется следующая величина:
	$$
	h(X, d, \mu) = \lim_{R \rightarrow \infty} \frac{\ln\wt{\mu}\left(B_{\wt{x}_0}^{\wt{X}}(R)\right)} {R}.
	$$
\end{defin}


\begin{remark}
Объёмная энтропия не зависит от выбора точки $\wt{x}_0$.
\end{remark}
\begin{proof}
Действительно, пусть мы выберем другую точку $\wt{x}^1_0$, $D = dist(\wt{x}_0, \wt{x}_0^1)$.
Очевидно, что $\ln\wt{\mu}\left(B_{\wt{x}_0}^{\wt{X}}(R)\right) \leq 
\ln\wt{\mu}\left(B_{\wt{x}_0^1}^{\wt{X}}(R + D)\right)$. Из этого неравенства и аналогичного ему, сразу видно, 
что на объёмную энтропию выбор точки $\wt{x}_0$ не влияет.
\end{proof}

\begin{remark}
Если метрику $d$ изменить в $c$ раз (то есть все расстояния умножатся на $c$), то объёмная энтропия поделится на $c$.
\end{remark}
\begin{proof}
Легко понять из определения.
\end{proof}

Несмотря на то, что мы поняли, что от выбора точки $\wt{x}_0$ объёмная энтропия не зависит, 
то, что предел существует, пока является сомнительным. В дальнейшем мы покажем, что он существует и ограничен.

Заметим, что так как $X$ -- компакт, то существует $R_0$ такое, что $d(x_1, x_2) \leq R_0$ для всех $x_1, x_2$.
Пусть $G = \pi_1(X, x_0)$ -- фундаментальная группа петель пространства $X$, а $<g_1,\dots,g_s>$ -- образующие.
Так как $X$ -- локально стягиваемое, то существует $r_0$ такое, что длина любой петли больше~$r_0$. 
Иными словами, $B_{r_0}^X(x_0)$ эквивалентен обычному шару.

Пусть $\wt{x}_0^i$ и $\wt{x}_0^j$ - две любые центральные точки.
Заметим, что по построению метрики $\wt{d}$ и величины $r_0$, $\wt{d}(\wt{x}_0^i, \wt{x}_0^j) \leq r_0$, т.е. шары
с центрами в этих точках и радиусами $\frac{r_0}{2}$ не пересекаются.
Аналогично получим, что $p \left(B_{\wt{x}_0}^{\wt{X}} (R_0) \right)$ есть всё $X$.



Определим следующие функции:

$$
F(R)=\wt{\mu}\left(B_{\wt{x}_0}^{\wt{X}}(R)\right),
f(R)=\ln(F(R)).
$$
$K(R)$ -- количество центральных точек в шаре $B_{\wt{x}_0}^{\wt{X}} (R)$, а $k(R) = \ln(K(R))$.

\begin{lemma}\label{l_K_1}
Для любых $R_1, R_2 > 0$, $K(R_1 + R_2) \leq K(R_1+R_0) \cdot K(R_2+R_0)$
\end{lemma}

\begin{proof}
Заметим, что все центральные точки $\wt{X}$ равносильны, т.е. если рассмотреть аналог функции $K(R)$, но с выделенной точкой не $\wt{x}_0$, 
а какой-то другой центральный, то мы получим точно такой же результат. Возьмём все центральные точки из шара $\wtB{R_1 + R_0}$, которых $K(R_1 + R_0)$,
и для каждой такой точки, возьмём все центральные точки на расстоянии не более $R_2 + R_0$. 
Очевидно, что мы получим не более, чем $K(R_1+R_0) \cdot K(R_2+R_0)$ точек. 

С другой стороны, если у нас есть центральная точка $\wt{x}_1$ на расстоянии не более $R_1 + R_2$ от $\wt{x}_0$, то мы можем взять сначала точку 
$\wt{y}$ такую, что $\wt{d}(\wt{y}, \wt{x}_0) \leq R_1 + \varepsilon$ и $\wt{d}(\wt{y}, \wt{x}_1) \leq R_2 + \varepsilon$,
а затем взять центральную $\wt{z}$ на расстоянии меньше $R_0$ от $\wt{y}$. 
Тогда получится, что $\wt{x}_1$ на расстоянии не более $R_2 + R_0$ от $\wt{z}$, а $\wt{z}$ на расстоянии не более $R_1 + R_0$ от $\wt{x}_0$.
Поскольку $\wt{x}_1$ была произвольной, мы доказали требуемое.
\end{proof}

\begin{lemma}\label{l_K_2}
Существует $C > 0$, такое, что $\forall R : |f(R) - k(R)| \leq C$.
\end{lemma}

\begin{proof}
Пусть $\mu_0 =  \wt{\mu}\left( B_{\wt{x}_0}^{\wt{X}}(\frac{r_0}{2}) \right)$, 
$M_0 =  \wt{\mu}\left( B_{\wt{x}_0}^{\wt{X}}(R_0) \right)$.\\
Возьмём все центральные точки, лежащие в $B_{\wt{x}_0}^{\wt{X}}(R + R_0)$ 
и объединим круги с центрами в них и радиуса $R_0$. Легко понять, что мы покроем $B_{\wt{x}_0}^{\wt{X}}(R)$, откуда получим, 
что $F(R) \leq M_0\cdot K(R + R_0)$. Остаётся заметить, что $K(R + R_0) \leq K(3 R_0) \cdot K(R)$, что прямо получается из леммы \ref{l_K_1}. 

Итак, $F(R) \leq M_0\cdot K(3 R_0) \cdot K(R)$, аналогично $F(R) \geq \frac{\mu_0}{K(3 R_0)}\cdot K(R)$. 
После взятия логарифма получаем требуемое.
\end{proof}

Используя лемму \ref{l_K_2}, мы получаем, что 

$$
	h(X, d, \mu) = \lim_{R \rightarrow \infty} \frac{k(R)} {R} = \lim_{R \rightarrow \infty} \frac{\ln (K(R))} {R},
$$
Откуда видно, что $h(X, d, \mu)$ не зависит от меры $\mu$, а зависит только от самого пространства, и той метрики, которая на нём задана.
Поэтому отныне мы будем обозначать объёмную энтропию как $h(X, d)$.
Тем не менее, нам ещё предстоит доказать, что она существует.

Применим несколько раз лемму \ref{l_K_1}, получим:
$$
  K(R_1) \cdot K(R_2) \geq K(R_1 + R_2 - 2 R_0) \geq \frac{K(R_1 + R_2)}{K(4 R_0)}.
$$
А из этого видим, что $k(R_1 + R_2) \leq k(R_1) + k(R_2) + C$, где $C = ln(K(4 R_0))$.
Заметим, что $\wt{k}(R) = k(R) + C$ удовлетворяет неравенству $\wt{k}(R_1 + R_2) \leq \wt{k}(R_1) + \wt{k}(R_2)$, и при этом,
$h(X,d) = \lim_{R \rightarrow \infty} \frac{k(R)}{R} = \lim_{R \rightarrow \infty} \frac{\wt{k}(R)}{R}$.
Тем самым, нам осталось доказать, что существует предел $\lim_{R \rightarrow \infty} \frac{\wt{k}(R)}{R}$.
Заметим, что $\frac{\wt{k}(2R)}{2R} \leq \frac{2\wt{k}(R)}{2R} = \frac{\wt{k}(R)}{R}$, поэтому величина $\frac{\wt{k}(R)}{R}$ ограничена сверху.
Пусть $A$ нижняя граница $\frac{\wt{k}(R)}{R}$. Мы уже поняли, что $A < \infty$, докажем, что $\lim_{R \rightarrow \infty} \frac{\wt{k}(R)}{R} = A$.

Возьмём $\eps > 0$. По определению $A$, есть $R^1$ такое, что $\frac{\wt{k}(R^1)}{R^1} < A + \eps$. 
Возьмём произвольное $R$ и разложим его в сумму $R = m R^1 + D$, где $m$ - натуральное, а $0 \leq D < R^1$. 
Заметим следующее:
$$
\frac{\wt{k}(R)}{R} \leq 
\frac{m \wt{k}(R^1) + \wt{k}(D)}{m R^1 + D} =
\frac{\wt{k}(R^1) + \frac{\wt{k}(D)}{m}}{R^1 + \frac{D}{m}} .
$$
Из неравенств выше следует, что при больших $R$(и соответственно больших $m$) $\frac{\wt{k}(R)}{R} \leq A + \eps$. 
Поскольку $A$ - нижняя граница, а $\eps$ - произвольный, то $\lim_{R \rightarrow \infty} \frac{\wt{k}(R)}{R}$ существует и равен $A$.




\section{Фундаментальная группа}
Напомним, что $G = \pi_1(X, x_0)$ -- фундаментальная группа петель пространства $X$, а $<g_1,\dots,g_s>$ -- образующие.
Пусть $d_1$ -- инфимум длин петель, гомотопных петле $g_1$. Аналогично определим $d_2,\dots,d_s$.
Положим $D_0 = \max(d_i)$.
\begin{defin}
Пусть $g \in G$. $\lambda(g)$ -- это количество $g_i$ в минимальном разложении $g$ по образующим.
\end{defin}
Заметим, что каждой центральной точке из $\wt{X}$ соответствует какая-то петля из $G$, а именно $\prod g_{k_i}$ 
причём расстояние от этой центральной точки до $\wt{x}_0$ не превосходит суммы соответствующих $d_{k_i}$. 
Тем самым мы доказали, что если некой центральной точке соответствует петля $g$,
$\lambda(g)$ при этом не больше $s$, то тогда расстояние до $\wt{x}_0$ от этой центральной точки не превосходит $D_0 \cdot s$.
Попробуем дать оценку в другую сторону.

\begin{defin}
	$L(k)$ - инфимум длин петель, минимальное разложение которых по образующим $g_i$ включает хотя-бы $k$ элементов.
\end{defin}
В дальнейшем мы будем рассматривать не сами петли, а их поднятие в пространство $\wt{X}$.

\begin{lemma}\label{l_C}
Существует $C > 0$, такое, что $L(k) \geq C \cdot k$
\end{lemma}
\begin{proof}
Для начала заметим, что $L(k)$ - неубывающая функция. Более того, ясно, что она неограниченная, так как если она ограниченна числом $R$, 
то тогда $K(R)$ есть $+\infty$, чего быть не может. Итак, пусть $k_0$ таково, что $L(k_0) > 3 R_0$.
Далее мы будем считать, что $k > k_0$. Ясно, что если сможем найти $C$ для таких $k$, то из этого будет следовать лемма.

Итак, пусть у нас есть произвольная петля $q_0$, $\lambda(q_0) \geq k$, а её поднятие -- путь $x_0 x_1$.
Отступим от точки $x_1$ по пути на $2R_0$, и получим точку $y$.
Мы сможем это сделать, так как длина пути больше $L(k)$, которое больше $3R_0$, по нашей договорённости.
Далее, около точки $y$ мы сможем найти центральную точку $z$ на расстоянии не более $R_0$, по построению $R_0$.
Рассмотрим проекцию путей $x_0yz$ и $zyx_1$ - это петли $q_1$ и $q_2$ соответственно, причём $q_1q_2 = q_0$.
Заметим, что $\lambda(q_1) + \lambda(q_2) \geq k$, а $\lambda(q_2) < k_0$, так как длина петли $q_2$ меньше $3R_0$. 

Из вышесказанного следует, что длина петли $q_1$ хотя бы $L(\lambda(q_1)) \geq L(k-k_0)$.
В свою очередь $L(k)$ хотя бы длина $x_0x_1$, которую можно оценить как длина $q_1$ $+ R_0$. 
Тем самым, мы доказали, что $L(k) \geq L(k-k_0) + R_0$, откуда легко получаем требуемое.
\end{proof}


\begin{theorem}
Пусть $\Lambda(k)$ -- количество петель $g$, для которых $\lambda(g) \leq k$. 
Тогда $h(X, d) > 0$ тогда и только тогда, когда $\Lambda(k)$ имеет экспонициальный рост, 
т.е. существует $c$ такое, что $\Lambda(k) > e^{ck}$.
\end{theorem}

\begin{proof}
Заметим, что $K(D_0 k) \geq \Lambda(k)$, так как длина петли $g$, такой, что $\lambda(g) \leq k$, не превосходит $D_0 k$.
Аналогично из леммы \ref{l_C} получаем, что $\Lambda(\frac{R}{C}) \geq K(R)$.

$C$ и $D_0$ постоянны, поэтому $\Lambda(k)$ имеет экспонициальный рост тогда и только тогда, когда $K(R)$ имеет экспонициальный рост,
что в свою очередь равносильно тому, что $h(X, d) > 0$.

\end{proof}

\begin{defin}
Ряд Пуанкаре:
$$
  F(t) = \sum_{\wt{y} \in p^{-1}(x_0)} e^{-t \cdot dist(\wt{y}, \wt{x}_0)}
$$
\end{defin}

\begin{theorem}\label{t_puank}
$$
h(X,d) = \inf \{t>0: F(t) < \infty\}
$$
\end{theorem}

\begin{proof} Мы докажем только то, что $h(X,d) \geq \inf \{t>0: F(t) < \infty\}$.

Пусть $c = h(X, d)$. Поскольку $\lim_{R \rightarrow \infty} \frac{\ln (K(R))} {R} = c$, то для любого $\eps > 0$ 
существует $R_1$, такое, что для любого $R > R_1$, $\left|\frac{\ln (K(R))} {R} - c\right| < \eps$.
Пусть $\frac{\ln (K(R))} {R} = c + \eps_R$, тогда мы знаем, что $|\eps_R| < \eps$.

Наша цель доказать, что если в ряд Пуанкаре подставить $c + 2\eps$, то он сойдётся.
Разобъём этот ряд на куски $A_N$, в которых $N \leq dist(\wt{y}, \wt{x}_0) < N+1$. На каждом из $A_N$ оценим $dist(\wt{y}, \wt{x}_0)$, как $N$, 
от чего слагаемые могут лишь увеличиться. Итак, получим такое соотношение:
$$
F(t) \leq \sum_{N=0}^{\infty} (K(N+1)-K(N))\cdot e^{-Nt}
$$
Ясно, что первые несколько слагаемых этого ряда не влияют на сходимость, поэтому мы будем рассматривать лишь слагаемые, в которых $N > R_1$.
Распишем такое слагаемое подробнее, и подставим $t = c + 2\eps$:
$$
(K(N+1)-K(N))\cdot e^{-Nt} = \frac{e^{(c + \eps_{N+1})(N+1)} - e^{(c + \eps_{N})N}}  {e^{(c+2\eps)N}} = 
$$
$$
= \frac{e^{(c + \eps_{N+1})(N+1)} - e^{(c + \eps_{N})N}}  {e^{(c+\eps)N}} \cdot e^{-\eps N}
$$
Заметим, что то, что стоит перед $e^{-\eps N}$ ограничено для любого N, т.к. $|\eps_{N+1}|, |\eps_{N}| < \eps$. 
Тем самым, мы показали, что ряд $F(c+2\eps)$ сходится.





\end{proof}

\section{Задачи}
Итак, мы видим, что фундаментальная группа тесно связана с объёмной энтропией, поэтому предлагается изучить, что будет, 
если в качестве пространства X взять граф, метрика на котором будет задаваться числами на рёбрах, 
т.е. ребро будет топологически эквивалентно отрезку, 
длина которого будет равна числу, указанному на нём. 
Топология задаётся естественным образом, а расстояние между произвольными точками как инфимум расстояний.

Мы будем задаваться следующим вопросом: где достигается минимум объёмной энтропии, при фиксированном графе и общей сумме чисел на рёбрах.
Для нахождения объёмной энтропии мы будем пользоваться рядом Пуанкаре и теоремой \ref{t_puank}.

\begin{problem}\label{z_first}
Дан произвольный граф. Сумма чисел на рёбрах равна $C$. Достигается ли минимум объёмной энтропии и отделён ли он от нуля?
\end{problem}
Пусть в этом графе $k$ ребер. Рассмотрим следующее подпространство $\mathbb{R}^k$:
$$
x_1 + x_2 + \dots + x_k = C, x_i \geq 0.
$$
На нём задана непрерывная функция, а именно объёмная энтропия графа $G$, 
с раставленными на рёбрах числами $x_1, x_2, \dots,x_k$.
Ясно, что это компактное подпространство $\mathbb{R}^k$, поэтому минимум достигается.
Остаётся понять, отделён ли он от нуля.

Итак, пусть минимум достигается на графе $H$. Ясно, что если в графе $H$ не более одного цикла, то объёмная энтропия такого графа равна нулю.
Докажем, что если там есть хотя бы два цикла, то тогда объёмная энтропия больше нуля.
Заметим, что если мы из графа $H$ выкинем несколько рёбер, то от этого объёмная энтропия только уменьшится. 
Тем самым мо можем легко привести граф к одному из следующих: 
две вершины, соединённые тремя кратными рёбрами; 
две вершини, соединённые ребром, и по одной петле на каждой;
одна вершина и две петли на ней.

А для этих графов лекго понять, что объёмная энтропия больше нуля.



\begin{problem}\label{z_two_circ} 
Пусть граф представляет из себя одну вершину и две петли длин $a$ и $b$, причём сумма фиксирована. 
При каких $a$ и $b$ достигается минимум объёмной энтропии?
\end{problem}

Давайте посмотрим на то, какие слагаемые есть в ряде Пуанкаре. А именно, там будут слагаемые такого типа: $e^{-(c_1 a + c_2 b)t}$, 
соответствующие всем петлям, которые $c_1$ раз проходят по ребру $a$ и $c_2$ раз проходят по ребру $b$. 
Бесспорно, на каждое слагаемое $e^{-(c_1 a + c_2 b)t}$ найдётся парное ему $e^{-(c_2 a + c_1 b)t}$, если $c_1$ и $c_2$ не равны.
Поэтому, если мы заменим $a$ и $b$ на их среднее арифметическое, то те слагаемые, в которых $c_1$ равно $c_2$ просто не изменятся,
а остальные -- уменьшатся, благодаря неравенству 
$$
e^{-(c_1 a + c_2 b)t} + e^{-(c_2 a + c_1 b)t} \geq 2 \cdot e^{-(c_1 + c_2)\frac{a + b}{2}t}.
$$
Мы видим, что от такой замены, объёмная энтропия не увеличилась. 
Тем самым ответ: при $a = b$ достигается минимум энтропии. Давайте его вычислим.
$$
F(t) = 1 + \sum_{k=1}^{\infty} 4 \cdot 3^{k-1} \cdot e^{-ka t} 
$$
Чтобы $F(t)$ сходился, необходимо, чтобы $3\cdot e^{-at} < 1$, т.е. ответ: $\frac{\ln(3)}{a}$.

\bigskip
Попробуем обобщить рассуждения из задачи \ref{z_two_circ}. 
Пусть даны графы $G_1,G_2,\dots,G_n$ с одинаковой структурой, но разными числами на рёбрах. 
При этом общая сумма чисел на рёбрах у всех графах одинаковая.

Зададим граф $G$: структура графа будет такая же, как у данных графов, а числа на рёбрах зададим как 
среднее арифметическое чисел на соответствующих рёбрах из графов $G_1, G_2, \dots, G_n$.

\begin{theorem}\label{t_srar}
$ h(G) \leq \max(h(G_i))$
\end{theorem}
\begin{proof}
Пусть $H = \max(h(G_i)) + \eps$, где $\eps > 0$. 
Докажем, что ряд Пуанкаре для графа $G$ сходится при $t=H$. Из этого будет следовать теорема.

Ясно, что ряды Пуанкаре для графов $G_1,\dots,G_n$ сходятся при $t=H$, по построению $H$.
Заметим, что:
$$
\frac{\sum_{i=1}^n e^{-t \cdot dist_{G_i}(\wt{y}, \wt{x}_0)}}{n} \geq 
e^{-t \cdot \frac{\sum_{i=1}^n dist_{G_i}(\wt{y}, \wt{x}_0)}{n} } = 
e^{-t \cdot dist_{G}(\wt{y}, \wt{x}_0)},
$$
где $\wt{y}$ - некая центральная точка, {\it одинаковая} для всех графов $G_i$ и $G$, но расстояния в каждом графе свои.
Последнее равенство верно из тех соображений, что расстояния в графе $G$ заданы как среднее арифметическое расстояний графов $G_1,\dots, G_n$.
Из соображений выше следует, что ряд Пуанкаре для графа $G$ не превосходит среднего арифметического рядов для $G_1, \dots, G_n$, 
откуда ясно, что он тоже сходится при $t=H$.
\end{proof}

Из этой теоремы теперь можно легко получить общий способ нахождения расстановки чисел на рёбрах, 
для которой достигается минимум объёмной энтропии.

\begin{problem} 
Тоже самое, что и в задаче \ref{z_two_circ}, но петель не две, а $n$.
\end{problem}

Пусть у нас есть некоторая расстановка чисел на этих ребрах. 
Зададим графы $G_1, G_2, \dots, G_{n!}$ как все возможные перестановки чисел на рёбрах.
Ясно, что объёмная энтропия у всех этих графов одинаковая.
Применив теорему \ref{t_srar} мы получим, что объёмная энтропия исходного графа больше или равна энтропии графа, 
в котором на рёбрах стоят одинаковые числа.

Итак, мы доказали, что минимум достигается, когда все рёбра имеют одинаковый вес $a$. 
Вычислим в таком случаи объёмную энтропию.
$$
F(t) = 1 + \sum_{k = 1}^{\infty} 2n\cdot (2n-1)^{k-1} e^{-akt}
$$
Тем самым, ответ: $\frac{\ln(2n-1)}{a}$. Или, если сумма весов 1, то $n\cdot \ln(2n-1)$.


\begin{problem}
То же самое, что и в предыдущих задачах, но в качестве графа взят полный граф на $n$ вершинах.
\end{problem}
Привлекая те же рассуждения, что и в предыдущей задаче, и не забывая факт про то, 
что объёмная энтропия не зависит от выбора выделенной точки $x_0$ получим, что минимум достигается, когда все рёбра равны.

\medskip
Обобщим все эти задачи.

\begin{theorem}
Дан реберно-однородный граф G. При фиксированной сумме чисел на рёбрах минимум объёмной энтропии достигается, когда все они равны.
\end{theorem}

\begin{proof}
Мы уже знаем, что минимум достигается при какой-то расстановке чисел. 
Пусть минимальное число на ребре равно $a$, а максимальное $b$. 
Рассмотрим графы $G_1, G_2, \dots, G_k$ - все возможные самосовмещения графа $G$ с самим собой.
Заметим, что объёмные энтропии всех этих графов одинаковы, так как единственное их отличие между собой -- 
разный выбор точки $x_0$. Применяя теорему \ref{t_srar} построим граф $G^1$. 
Заметим, что все рёбра нового графа имеют длину хотя бы $a + \frac{b-a}{k}$, 
так как длина ребра вычислялась как средлее арифметическое соответствующих рёбер $G_i$, и, 
так как граф $G$ -- реберно однородный, то хотя бы раз встречалось ребро длины $b$.
Аналогично длины всех рёбер $G^1$ не превосходят $b-\frac{b-a}{k}$.

Таким образом, мы можем построить последовательность графов $G^1,G^2,\dots$, которые реализуют минимальную энтропию, 
и при этом числа на рёбрах этого графа стремятся к среднему значению. 
Тем самым граф, у которого все рёбра одинаковой длины тоже реализует минимум объёмной энтропии.

\end{proof}






\end{document}
